% !TEX TS-program = LuaLaTeX
% Exemple d'ordre du jour
% par Pierre Tremblay, Universite Laval
% modifie par Christian Gagne, Universite Laval
% 2011/01/14 - version 1.4
% modifié par Robert Bergevin, Université Laval
% 24/11/2011
% modifié par Jean-Yves Chouinard, Université Laval
% 2016/01/11
% modifié par Jean-Yves Chouinard, Université Laval
% 2017/01/04
%

%--------------------------------------------------------------------------------------
%------------------------------------- preambule --------------------------------------
%--------------------------------------------------------------------------------------
\documentclass[12pt]{../ULojpv}

% Chargement des packages supplementaires
\usepackage[utf8]{inputenc}
\usepackage{hyperref}
\usepackage{import}
\hypersetup{
    colorlinks=true,
    linkcolor=blue,
    filecolor=magenta,      
    urlcolor=cyan,
}
% Definitions des parametres de l'en-tete
\Cours{GEL -- 1001 Design I (méthodologie)}             % Nom du cours
\NumeroEquipe{04}                                     % Numero de l'equipe
\NomEquipe{SécuritAIR}                               % Nom de l'equipe
\Objet{Ordre du jour}                                 % Nom du document
\SujetRencontre{Organisation et planification}        % Sujet de la rencontre
\DateRencontre{2020/01/28}                            % Date de la rencontre
\LocalRencontre{Teams}                            % Local de la rencontre
\HeureRencontre{13h30-15h00}                          % Heure de la rencontre

%--------------------------------------------------------------------------------------
%--------------------------------- corps du document ----------------------------------
%--------------------------------------------------------------------------------------
\begin{document}
\entete
\begin{enumerate}
   \item \textbf{Ouverture de la réunion}
   \item \textbf{Nomination ou confirmation du président et du secrétaire}
   \item \textbf{Lecture et adoption de l'ordre du jour}
   \item \textbf{Lecture et adoption du procès-verbal de la réunion du 12 janvier 2017}
   \item \textbf{Affaires découlant du procès-verbal}
      \begin{enumerate}
         \item Affaire \#1
            \begin{enumerate}
               \item Sous-affaire \#1
               \item Sous-affaire \#2
            \end{enumerate}
         \item Affaire \#2
      \end{enumerate}
   \item \textbf{Points à traiter}
   \begin{enumerate}
      \item \import{./membres/}{etienne.tex} 
      \item \import{./membres/}{gael.tex} 
      \item \import{./membres/}{jonathan.tex} 
      \item \import{./membres/}{francis.tex} 
      \item \import{./membres/}{charles.tex} 
      \item \import{./membres/}{philippe.tex} 
      \item \import{./membres/}{yasser.tex} 
      \item \import{./membres/}{thomas.tex} 
   \end{enumerate}
   \item \textbf{Divers}
   \item \textbf{Répartition des tâches}
   \item \textbf{Évaluation de la réunion}
   \item \textbf{Date, heure, lieu et objectif de la prochaine réunion}
   \item \textbf{Fermeture de la réunion}
\end{enumerate}

\end{document}

